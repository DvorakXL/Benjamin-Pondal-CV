\documentclass[10pt,a4paper,sans]{moderncv}

\moderncvstyle{casual}

\moderncvcolor{blue}

\usepackage[scale=0.80]{geometry}
\setlength{\hintscolumnwidth}{3.00cm}

\usepackage[utf8]{inputenc}
\usepackage[spanish]{babel}

\name{Benjamín}{Pondal}

\begin{document}

\maketitle

\section{\textsc{Información Personal}}
\cvitem{Residencia}{Martínez, Buenos Aires}
\cvitem{Teléfono Móvil}{+54 11 4038-7105}
\cvitem{Email}{benjaminpondal@hotmail.com}
\cvitem{Perfil de GitHub}{\url{https://github.com/DvorakXL}}

\section{\textsc{Educación}}
\cventry{2009 - 2021}{Bachiller en Economía y Administración}{Colegio San Gabriel}{}{}{
    \textbf{Promedio General} 8.50
}
\cventry{2022 - presente}{Licenciatura en Ciencias de la
    Computación}{Universidad de Buenos Aires}{Facultad de Ciencias Exactas y Naturales}{}{
    \textbf{Fecha prevista de graduación} Diciembre 2028
}
\section{\textsc{Experiencia}}
\cventry{Enero 2025}{Desarrollador Full Stack}{Renvance}{Startup}{}{
    \begin{itemize}
        \item \textbf{Backend:} Diseño de REST APIs con Python y gestión de servicios AWS (Lambda, S3).
        \item \textbf{Frontend:} Creación de componentes interactivos en React.js con integración de estados vía Redux.
        \item \textbf{Testing:} Pruebas end-to-end utilizando Cypress.
    \end{itemize}
    \textbf{Contacto de referencia:} \href{mailto:alejandro.mildiner@renvance.com}{alejandro.mildiner@renvance.com}
}
\cventry{2023 - Presente}{Hacking Ético y Ciberseguridad}{HackerOne}{}{}{
    Identificación y reporte de vulnerabilidades en plataformas web:
    \begin{itemize}
        \item Vulnerabilidad Cross Site Scripting de alta severidad en Reddit (red social utilizada por más de 70 millones de usuarios activos diarios).
              \newline \url{https://hackerone.com/reports/1962645}
        \item Vulnerabilidad Cross Site Scripting de severidad media en Mail.ru (servicio de email ruso utilizado por mas de 30 millones de usuarios).
              \newline \url{https://hackerone.com/reports/1121980}
    \end{itemize}
    \textbf{Enfoque técnico:} Reconocimiento con Burp Suite, explotación manual y automatizada utilizando Kali Linux, redacción de PoCs.
}

% \cvitem{Hacking Ético y Ciberseguridad}{
%     Vulnerabilidad Cross Site Scripting de alta severidad en Reddit (red social utilizada por más de 70 millones de usuarios activos diarios).
%     \newline \url{https://hackerone.com/reports/1962645} \newline
%     Vulnerabilidad Cross Site Scripting de severidad media en Mail.ru (servicio de email ruso utilizado por mas de 30 millones de usuarios).
%     \newline \url{https://hackerone.com/reports/1121980}
% }
% \cvitem{Desarrollo Full Stack}{
%     Desarrollo independiente de una red social para publicar fotos y videos. Tecnologias utilizadas: ReactJS, CSS, HTML (frontend), Express y MySQL (backend). \newline
%     \url{https://github.com/DvorakXL/Tik-Pic}
% }

\section{\textsc{Proyectos}}
\cvlistitem{
    Desarrollo de Mods para Minecraft (Java - Forge API). Uso de programación orientada a objetos para extender las clases base del juego y añadir nuevas mecánicas.
}
\cvlistitem{Desarrollo de un Bot para la plataforma Discord utilizando NodeJS y consumiendo la API de Steam. Servido localmente y brindando serivicios a más de 10 servidores privados.}
\cvlistitem{Desarrollo de videojuegos en Unity y Godot implementando patrones de diseño para Programación Orientada a Objetos como Finite State Machines, Model View Controller y Factory Pattern.}

\section{\textsc{Aptitudes Técnicas}}
\cvitem{Experiencia básica}{C\#, Java, TypeScript, Python, Unity, ReactJS, HTML, CSS, MySQL, AWS, Cypress, Unit Testing y Git.}

\section{\textsc{Idiomas}}
\cvdoubleitem{Inglés}{Nivel avanzado. Cambridge English: First (FCE)}{Español}{Hablante nativo.}

\end{document}

